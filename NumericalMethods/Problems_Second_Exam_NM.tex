\documentclass[12pt]{article}
\usepackage[T1,T2A]{fontenc}
\usepackage[utf8]{inputenc}
\usepackage[bulgarian]{babel}
\usepackage{amssymb,amsmath,tabu}
\usepackage{amsthm}
\usepackage{tikz}
\usepackage{graphicx}
\usepackage{tcolorbox}
\usepackage{pgfplots}

\pgfplotsset{width=7cm, compat=1.10}
\usepgfplotslibrary{fillbetween}
\pgfmathdeclarefunction{poly}{0}{\pgfmathparse{3*x^2-1}}
\pgfmathdeclarefunction{origpoly}{0}{\pgfmathparse{x^3-x}}

\def\b#1{\mathbf{#1}}
\def\r#1{\mathrm{#1}}
\newcounter{problem}
\newcounter{solution}
\setcounter{problem}{0}%
\newcommand\problem{%
  \stepcounter{problem}%
  \textbf{Задача \theproblem.}~%
  \\
}

\newcommand\solution{%
  \textbf{Решение:}\\~%
}

\begin{document}
\begin{titlepage}
\begin{center}
    \vspace{5em}
    \textbf{\Huge{Задачи второ контролно ЧМ}}\\
    \vspace{2em}
    \LARGE{Информатика при Лозко Милев}
\end{center}
\vfill

Теодора Иванова \hfill 20.05.2019
\end{titlepage}
\begin{center}{\section*{Първи тип}}\end{center}
\problem
Като използвате интерполационна формула на Лагранж, намерете полинома $p\in\pi_2$, който удовлетворява условията: $p(-1) = 2, p(1)  = 2, p(2) = 5$. Представете $p(x)$ по степените на $x$.
\solution
Използваме формулата
\begin{equation*}
L_2(f;x) = \sum_{k=0}^{2}f(x_k)\prod_{i=0,i\neq k}^{2}\frac{x-x_i}{x_k-x_i}
\end{equation*}
за $x_0=-1, x_1 = 1, x_2 = 2$
и получаваме
\begin{align*}
L_2(f;x) &= f(-1)\frac{(x-x_1)(x-x_2)}{(x_0-x_1)(x_0-x_2)}+f(1)\frac{(x-x_0)(x-x_2)}{(x_1-x_0)(x_1-x_2)}+f(2)\frac{(x-x_0)(x-x_1)}{(x_2-x_0)(x_2-x_1)}\\
&=2\frac{(x-1)(x-2)}{(-1-1)(-1-2)}+2\frac{(x+1)(x-2)}{(1+1)(1-2)}+5\frac{(x+1)(x-1)}{(2+1)(2-1)}\\
&=2\frac{x^2-3x+2}{6}+2\frac{x^2-x-2}{-2}+5\frac{x^2-1}{3}\\
&=\frac{x^2-3x+2+5x^2-5}{3} - x^2+x+2 = \frac{6x^2-3x-3-3x^2+3x+6}{3}\\
&=\frac{3x^2+3}{3} = \boxed{x^2+1}
\end{align*}
\problem
Полиномът $L_2(f;x)$ интерполира $f(x)= e^x$ в $-1, 0, 1$. Като използвате формулата за оценка на грешката докажете, че:
\begin{equation*}
\underset{x\in[-1,1]}{max}|f(x)-L_2(f;x)|\leq \frac{1}{5}.
\end{equation*}
\solution
Формулата за оценка на грешката има вида:
\begin{equation*}
f(x)-L_n(f;x) = \frac{f^{(n+1)}(\xi)}{(n+1)!}w(x), \xi\in[-1,1].
\end{equation*}
Замествайки за $x_0=-1, x_1 = 0, x_2 = 1$ получаваме:
\begin{align*}
f(x)-L_2(f;x) &= \frac{f^{3}(\xi)}{(3)!}(x-x_0)(x-x_1)(x-x_2) \\
&=\frac{e^\xi}{6}(x+1)x(x-1) = \boxed{\frac{e^\xi}{6}(x^3-x).}
\end{align*}
Търсим максимума на получената функция
\begin{equation*}
(f(x)-L_2(f;x))'=\frac{e^\xi}{6}(3x^2-1)
\end{equation*}
Този полином се нулира в $x=\pm\frac{1}{\sqrt{3}}$
% сложи картинка

\begin{tikzpicture}
  \begin{axis}[
    axis y line = center,
    axis x line = center,
    xtick       = {-1,-0.5773502691896258,0,0.5773502691896258,1},
    xticklabels = {$-1$,$\frac{-1}{\sqrt{3}}$,$0$,$\frac{1}{\sqrt{3}}$,$1$},
    ytick       = {3},
    yticklabels = {$f(x)$},
    samples     = 160,
    domain      = -1:1,
    xmin = -1.5, xmax = 1.5,
    ymin = -5, ymax = 5,
  ]
  \addplot[name path=poly, black, thick, mark=none, ] {poly};
  % \addplot[name path=poly, black, thick, mark=none, ] {origpoly};
\end{axis}
\end{tikzpicture}

Като максимума си достига при $x=\frac{-1}{\sqrt3}$.\\
От $\xi\in[-1,1]$ следва, че $e^\xi$ достига максимума си при $x=1$.
Тогава
\begin{align*}
\underset{x\in[-1,1]}{max}|f(x)-L_2(f;x)| \leq \frac{e}{6}\underset{x\in[-1,1]}{max}\left |\frac{-2}{3\sqrt3}\right|
\end{align*}
От $e < 3$ и $\sqrt3 < 2$
\begin{equation*}
\underset{x\in[-1,1]}{max}|f(x)-L_2(f;x)| \leq\frac{3}{6}\frac{2}{6}=\frac{1}{6} \leq \frac{1}{5} 
\end{equation*}
\problem
Като използвате интерполационна формула на Нютон с разделени разлики, намерете полинома $p\in\pi_3$, който удовлетворява условията: $p(-2) = -8, p(0) = 2, p(1)=4, p(2) = 12$.Представете $p(x)$ по степените на $x$.
\solution
Използваме таблицата за пресмятане на разделените разлики спрямо рекурентната им дефиниция.\\
\begin{tcolorbox}[colback=red!5!white,colframe=red!75!black]
    Нека $x_0, \dotso, x_n$ са дадени различни точки. Разделената разлика на функция $f$ в точките $x_0, \dotso, x_n$ се бележи с $f[x_0, \dotso, x_n]$ и се определя индуктивно със следната рекурентна връзка.
    \begin{equation*}
      f[x_0, \dotso, x_n] = \frac{f[x_1, \dotso, x_n] - f[x_0, \dotso, x_{n-1}]}{x_n - x_0},\hspace{12pt}n = 1, \dotso, \infty
    \end{equation*}
    , като приемаме, че $f[x_i] = f(x_i)$
\end{tcolorbox}
\begin{tabular}{|c c c c c|}
\hline
$x_i$&$f_i$&$f[.,.]$&$f[.,.,.]$&$f[.,.,.,.]$\\
\hline
$-2$ & $-8$ &     &      &\\
     &      & $5$ &      &\\
$0$  & $2$  &     & $-1$ &\\
     &      & $2$ &      &$1$\\
$1$  & $4$  &     & $3$  &\\
     &      & $8$ &      &\\
$2$  & $12$ &     &      &\\
\hline
\end{tabular}\\
Използваме формулата на Нютон
\begin{equation*}
L_n(f;x) = \sum_{k=0}^nf[x_0,\dotso,x_k](x-x_0)\dotso(x-x_{k-1}),
\end{equation*}
като приемаме, че $(x-x_0)\dotso(x-x_{k-1}) = 1$ при $k=0$.
\begin{align*}
p(x)=L_3(f;x) &= \sum_{k=0}^3f[-2,\dotso,2](x+2)\dotso(x-x_{k-1})\\
&=f[-2]+f[-2,0](x+2)x+f[-2,0,1](x+2)x+f[-2,0,1,2](x+2)x(x-1)\\
&= -8 + 5x+10-x^2-2x+x^3+x^2-2x \\
&= \boxed{2+x+x^3}
\end{align*}
\problem
Нека $S_k=1^2+\ldots+k^2$ за $k\geq1, S_0=0$. Покажете, че съществува единствен $p\in\pi_3:p(k)=S_k, k=0,1,2,\ldots$. Намерете $S_k$ като използвате формулата на Нютон с крайни разлики за интерполация напред.\\
\solution
\begin{tcolorbox}[colback=red!5!white,colframe=red!75!black]
Крайна разлика от $k$-ти ред дефинираме индуктивно по следния начин:
\begin{align*}
\Delta^0f_i&=f_i\\
\Delta^kf_i&=\Delta^{k-1}f_{i+1}-\Delta^{k-1}f_i
\end{align*}
\end{tcolorbox}
Използвайки дефиницията за крайни разлики ще покажем, че всички крайни разлики от 4-ти ред са 0.\\
\begin{tabular}{|c c c c c c|}
\hline
$k$&$S_k$&$\Delta S_k$&$\Delta^2S_k$&$\Delta^3S_k$&$\Delta^4S_k$\\
\hline
$0$ & $0$  &      &      &     &     \\
    &      & $1$  &      &     &     \\
$1$ & $1$  &      & $3$  &     &     \\
    &      & $4$  &      & $2$ &     \\
$2$ & $5$  &      & $5$  &     & $0$ \\
    &      & $9$  &      & $2$ &     \\
$3$ & $14$ &      & $7$  &     & $0$ \\
    &      & $16$ &      & $2$ &     \\
$4$ & $30$ &      & $9$  &     & $0$ \\
    &      & $25$ &      & $2$ &     \\
$5$ & $55$ &      & $11$ &     & $0$ \\
    &      & $36$ &      & $2$ &     \\
$6$ & $71$ &      & $13$ &     &     \\
....& .... &......&......&.....&.....\\
\hline
\end{tabular}\\
От това, че всички крайни разлики от 4-ти ред са 0, следва, че интерполационния полином на $S_k$ е от $\pi_3$.
За да намерим $p(x)\in\pi_3$ ще използваме формула на Нютон за интерполиране напред.
Провери формулите
\begin{tcolorbox}[colback=red!5!white,colframe=red!75!black]
Формула на Нютон за интерполиране напред в равноотдалечени възли
\begin{align*}
L_n(f;x_0+th)=\sum_{k=0}^{n}{t \choose k}\Delta^kf_0
\end{align*}
\end{tcolorbox}
\begin{tcolorbox}[colback=red!5!white,colframe=red!75!black]
КОМЕНТАР\\
Формула на Нютон за интерполиране назад в равноотдалечени възли
\begin{align*}
L_n(f;x_n+th)=\sum_{k=0}^{n}{t+k-1 \choose k}\Delta^kf_{n-k}
\end{align*}
\end{tcolorbox}
Тоест 
\begin{align*}
p(x)=L_n(S;x)&=\sum_{k=0}^{3}{x \choose k}\Delta^kS_0 =
 {x \choose 0}\Delta^0S_0+{x \choose 1}\Delta S_0+{x \choose 2}\Delta^2S_0+{x \choose 3}\Delta^3S_0 \\
&=0+x.1+\frac{x(x-1)}{2!}.3+\frac{x(x-1)(x-2)}{3!}.2 \\
&= x+\frac{3x^2-3x}{2}+\frac{2x^3-6x^2+4x}{6}=\frac{6x+9x^2-9x+2x^3-6x^2+4x}{6}\\&=\frac{2x^3+3x^2+x}{6} =\boxed{\frac{x^3}{3}+\frac{x^2}{2} +\frac{x}{6}}
\end{align*}
\problem
Като използвате интерполационна формула на Нютон с разделени разлики с кратни възли, намерете интерполационния полином на Ермит, който усовлетворява условията: $p(0) = -1, p'(0) = 1, p''(0)=2, p(1) =0,p'(1)=-1$. Представете $p(x)$ по степените на $x$.\\
\solution
\begin{tcolorbox}[colback=red!5!white,colframe=red!75!black]
Теорема.Нека $x_0\leq x_1\leq\ldots\leq x_n$ е редица от реални числа и $f(x)$ е достатъчно гладка функция в интервал, който ги съдържа. Тогава
\begin{align*}
f[x_0,x_1,\ldots,x_n] = \begin{cases}
                            \frac{f[x_1,\ldots,x_n]-f[x_0,\ldots,x_{n-1}]}{x_n-x_0},x_0<x_1<\ldots<x_n\\
                            \frac{f^{(n)}(x_0)}{n!} , x_0=x_1=\ldots=x_n
                        \end{cases}
\end{align*}
\end{tcolorbox}
Използвайки горната теорема ще пресметнем таблицата с разделените разлики:
\begin{tabular}{|c c c c c c|}
\hline
$x_i$&$f_i$&$f[.,.]$&$f[.,.,.]$&$f[.,.,.,.]$\\
\hline
$0$ & $-1$ &      &                  &      &\\
    &      & $1$  &                  &      &\\
$0$ & $-1$ &      & $\frac{2}{2!}$=1 &      &\\
    &      & $1$  &                  & $-1$ &\\
$0$ & $-1$ &      & $0$              &      & $-1$\\
    &      & $1$  &                  & $-2$ &\\
$1$ & $0$  &      & $-2$             &      &\\
    &      & $-1$ &                  &      &\\
$1$ & $0$  &      &                  &      &\\
\hline
\end{tabular}\\
Тогава интерполационния полином ще има вида:
\begin{align*}
p(x)=L_5(f;x) &= \sum_{k=0}^5f[0,\dotso,1]x\dotso(x-x_{k-1})\\
&= f[0] + f[0,0]x+f[0,0,0]x.x+f[0,0,0,1]x.x.x+f[0,0,0,1,1]x.x.x.(x-1)\\
&= -1 + x+ x^2 -x^3 -x^4 +x^3 = \boxed{-1 +x +x^2 -x^4}
\end{align*}
\problem
Като използвате формулата за тригонометрична интерползция при равноотдалечени възли, определете коефициентите $a_0, a_1, b_1$ така, че $\tau(x)= \frac{a_0}{2}+a_1cosx+b_1sinx$ да удовлетворява условията $\tau(0)=-1, \tau(\frac{2\pi}{3})=2,\tau(\frac{4\pi}{3})=2$.\\
\solution
\begin{tcolorbox}[colback=red!5!white,colframe=red!75!black]
Теорема.Нека $0\leq x_0<x_1<\ldots<x_{2n}<2\pi$ са дадени възли и $y_0,\ldots,y_{2n}$ са дадени числа. Тогава съществува единствен тригонометричен полином от степен $n - \tau_n(x):\tau_n(x_i)=y_i,i=0,\ldots,2n$, който се задава с формулата
\begin{align*}
\tau_n(x)&=\sum_{k=0}^{2n}\lambda_k(x)y_k,\text{ където}\\
\lambda_k(x)&=\prod_{i=0,i\neq k}^{2n}\frac{sin\left(\frac{x-x_i}{2}\right)}{sin\left(\frac{x_k-x_i}{2}\right)}
\end{align*}
\end{tcolorbox}

\problem
Да се намери явния вид на $B(1,2,4,t)$ за $t\in[1,4]$.\\
\solution
% Използваме основната рекурентна връзка:
% \begin{equation*}
% B_{i,r+1} = \frac{t-t_i}{t_{i+r}-t_i}B_{i,r}(t) + \frac{t_{i+r+1}-t}{t_{i+r}-t_{i+1}}B_{i+1,r}(t)
% \end{equation*}
% //виж как може да я ползваш при сметките
Използваме 
\begin{equation*}
B(t_0,t_1,t_2,t)=\begin{cases}
                   \frac{t-t_0}{(t_2-t_0)(t_1-t_0)},\hfill t\in[t_0,t_1]\\
                   \frac{t_2-t}{(t_2-t_0)(t_2-t_1)},\hfill t\in[t_1,t_2]\\
                   0\hfill,\text{ иначе}
                 \end{cases}.
\end{equation*}
И получаваме
\begin{equation*}
B(1,2,4,t)=\begin{cases}
             \frac{t-1}{(4-1)(2-1)},\hfill t\in[1,2]\\
             \frac{4-t}{(4-2)(4-1)},\hfill t\in[2,4]\\
             0\hfill,\text{ иначе}
           \end{cases} =
           \begin{cases}
             \frac{t-1}{3},\hfill t\in[1,2]\\
             \frac{4-t}{6},\hfill t\in[2,4]\\
             0\hfill,\text{ иначе}
           \end{cases}. 
\end{equation*}
\problem
Да се намери полинома на най-добро равномерно приближение от $\pi_1$ за функцията $f(x) = \left|x-\frac{1}{2}\right|$ в $[-1,1]$ и $E_1(f)$.\\
\solution
//Избираме средната точка да е там където полинома си сменя знака
Нека $-1, \frac{1}{2}, 1$ са точки на алтернанс, тоест
\begin{equation*}
E_1(f) = +[f(-1)-p(-1)]=-[f(\frac{1}{2})-p(\frac{1}{2})]=+[f(1)-p(1)]
\end{equation*}
и търсения полином е $p(x) = ax+b$.
//Направи го и за произволна точка t.
Тогава
\begin{align*}
\frac{3}{2} + a - b &= \frac{1}{2}a + b = \frac{1}{2} - a - b\\
\frac{3}{2} + a - b &=\frac{1}{2} - a - b\Rightarrow 2a = -1 
\Rightarrow \boxed{a = -\frac{1}{2}}\\
\frac{3}{2} + a - b &= \frac{1}{2}a + b \Rightarrow 2b = \frac{3}{2} + \frac{1}{2}a \Rightarrow \boxed{b = \frac{5}{8}}
\Rightarrow p(x) = \frac{1}{2}a-\frac{5}{8}
\end{align*}.
От $p(x)$ линейна функция, следва че в $[-1,1/2]$ тя достига своите екстремуми в краищата наинтервала и аналогично и за $[1/2,1]$.
Тогава $E_1(f) = \underbrace{\frac{3}{2} -\frac{1}{2} - \frac{5}{8}}_{\frac{3}{8}} = \underbrace{-\frac{1}{2}.\frac{1}{2} + \frac{5}{8}}_\frac{3}{8} = \underbrace{\frac{1}{2} + \frac{1}{2} - \frac{5}{8}}_\frac{3}{8}\Rightarrow E_1(f)=\frac{3}{8}$\\
\problem
Да се намери полинома на най-добро средноквадратично приближение от $\pi_1$ за функцията $f(x) = e^x$ в интервала $[-1,1]$ при тегло $\mu(x)\equiv1.$\\
\solution
Търсим $p=ax+b$, чиито базисни функции са $\varphi_0=1, \varphi_1=x$.
За да минимизираме
\begin{equation*}
\int_{-1}^{1}\mu(x)(f(x)-p(x))^2dx
\end{equation*}
Трябва да са изпълнени следните условия за ортогоналноста с $\pi_1$.
\begin{align*}
&\begin{array}{|c}
\int_{-1}^{1}\mu(x)(f(x)-p(x))dx = 0\\
\\
\int_{-1}^{1}\mu(x)(f(x)-p(x))xdx = 0
\end{array}
\Leftrightarrow
\begin{array}{|c}
\int_{-1}^{1}(e^x-ax - b)dx = 0\\
\int_{-1}^{1}(e^x -ac - b)xdx = 0 
\end{array}\\
\Rightarrow
&\begin{array}{|c}
\int_{-1}^{1}e^xdx = a\int_{-1}^{1}xdx+b\int_{-1}^{1}dx\\
\\
\int_{-1}^{1}e^xxdx = a\int_{-1}^{1}x^2dx+b\int_{-1}^{1}xdx
\end{array}
\Rightarrow
\begin{array}{|c}
e^x\rvert_{-1}^{1} = a\frac{x^2}{2}\rvert_{-1}^{1}+bx\rvert_{-1}^{1}\\
\\
xe^x\rvert_{-1}^{1} - e^x\rvert_{-1}^{1} = a\frac{x^3}{3}\rvert_{-1}^{1}xdx+b\frac{x^2}{2}\rvert_{-1}^{1}
\end{array}\\
\Rightarrow
&\begin{array}{|c}
e-\frac{1}{e}=2b\\
\\
e + \frac{1}{e} - e + \frac{1}{e} = \frac{2}{3}a
\end{array}
\Rightarrow
\boxed{\begin{array}{|c}
b = \frac{1}{2}\left(e-\frac{1}{e}\right)\\
\\
a = \frac{3}{e}
\end{array}}
\end{align*}
Тогава
\begin{equation*}
\boxed{p(x) = \frac{3}{e}x + \frac{e}{2} - \frac{1}{2e}}.
\end{equation*}
\problem
Да се намери полинома от $\pi_1$, приближаващ по метода на най-малките квадрати таблицата:
\begin{equation*}
\begin{array}{c|c|c|c|c|}
x_i & 0 & 1 & 3 & 4\\
\hline
f_i & 5 & 2 & 2 & 1
\end{array}
\end{equation*}
\solution
Търсим полинома $p(x) \in \pi_1:p(x) =ax+b$, където $\varphi_0(x) = x, \varphi_1(x) = 1$ са базисните полиноми, за който 
\begin{equation*}
\sum_{i=1}^{4}[f_i-p(x_i)]^2
\end{equation*}
е минимална.
От ортогоналността получаваме системата
\begin{align*}
&\begin{array}{|c}
\sum_{i=1}^{4}[f_i-p(x_i)]^2 = 0\\
\sum_{i=1}^{4}[f_i-p(x_i)]^2x=0
\end{array}
\Leftrightarrow
\begin{array}{|c}
\sum_{i=1}^{4}[f_i-ax_i-b] = 0\\
\sum_{i=1}^{4}[f_i-ax_i-b]x=0
\end{array}\\
\Rightarrow
&\begin{array}{|c}
\sum_{i=1}^{4}[f_i] =a\sum_{i=1}^{4}x_i+b\sum_{i=1}^{4}\\
\sum_{i=1}^{4}[f_i]x_i=a\sum_{i=1}^{4}x_i^2 + b\sum_{i=1}^{4}x_i
\end{array}
\Rightarrow
\begin{array}{|c}
10 = 8a + 4b\\
12 = 26a + 8b
\end{array}
\Rightarrow
\boxed{
\begin{array}{|c}
a = -\frac{4}{5}\\
b = \frac{41}{10}
\end{array}}
\end{align*}
Тогава $p(x) = -\frac{4}{5}x+\frac{41}{10}$.\\
\begin{center}{\section*{Втори тип}}\end{center}
\setcounter{problem}{6}
\problem
Нека $B(x_0,\ldots,x_r;t)$  е  В-сплайнът от степен $r-1$ с възли $x0<\ldots<x_r.$ Да се намери $\int_{x_0}^{x_r}B(x_0,\ldots,x_r,t)dt$. Отговорът да се представи като функция зависеща само от $r$.
\\
\solution
\begin{equation*}
B(x_0,\ldots,x_r;t)=(x-t)^{r-1}_+[x_0,\ldots,x_r]=\sum_{k=0}^{r}c_k(x_k-t)^{r-1}_+, \text{където} c_k = \frac{1}{w'(x_k)}
\end{equation*}.
От
\begin{equation*}
f[x_0,\ldots,x_r] = \sum_{k=0}^{r}c_kf(x_k).
\end{equation*}
Разглеждаме за някое $k$ примитивната
\begin{equation*}
\int (x_k-t)^{r-1}dt = \frac{(x_k-t)^r}{-r},
\end{equation*}
където $(x_k-t)<0$ и $-r<0$ тоест $\frac{(x_k-t)^r}{-r}<0$. Тогава
\begin{align*}
&\int_{x_0}^{x_r}B(x_0,\ldots,x_r;t)dt = 
\sum_{k=0}^{r}c_k\int_{x_0}^{x_r}(x_k-t)^{r-1}_+
=\left.\left(\frac{-1}{r}\sum_{k=0}^{r}x_k(x_k-t)^r_+\right)\right\rvert_{x_0}^{x_r} \\
&= -\frac{1}{r}(x-t)^r_+[x_0,\ldots,x_r]\rvert_{x_0}^{x_r} 
= -\frac{1}{r}((x-x_r)^r_+[x_0,\ldots,x_r] - (x-x_0)^r_+[x_0,\ldots,x_r]) \\
&= -\frac{1}{r}(0 - (x-x_0)^r_+[x_0,\ldots,x_r]) \\
&(x-x_0)^r_+[x_0,\ldots,x_r] - \text{коефициента пред } x^r = 1.
\end{align*}
Тогава $\boxed{\int_{x_0}^{x_r}B(x_0,\ldots,x_r;t)dt = \frac{1}{r}}$\\
\problem
Да се докаже, че най-доброто равномерно приближение с полиноми от $\pi_n$ за функцията $f(x) = cos\r x$ в $[-1,1]$ удовлетворява неравенството: $E_n(f) \leq \frac{1}{2^n(n+1)!}$.\\
\solution
\begin{equation*}
E_n(f) = \underset{p*\in\pi_n}{min\text{ }\rho(f;p*)}\leq \rho(f;p),\forall p\in\pi_n.
\end{equation*}
Нека $p$ е интерполационния полином на Лагранж във възлите $\xi_0,\ldots,\xi_n$, които са нулите на $T_{n+1} = cos\left((n+1) arccos \r x\right)$.
\begin{equation*}
\rho(f;p)=\underset{x\in [-1,1]}{max}|f(x)-p(x)|
\end{equation*}
\begin{equation*}
|f(x)-p(x)| = \left |\frac{f^{(n+1)}(\xi)}{(n+1)!}(x-\xi_0)\ldots(x-\xi_n)\right| \leq \frac{1}{(n+1)!}.\frac{1}{2^n}|T_{n+1}(x)|\leq\frac{1}{2^n (n+1)!}
\end{equation*}
От $T_{n+1}(x)=c(x-\xi_0)\ldots(x-\xi_n)$, следва че $|T_{n+1}|\leq 1$.\\
\problem
Докажете, че ако ако $f\in C^1[0,1]$ то за производната на полинома на Бернщайн е изпълнено:
\begin{equation*}
B'_{n+1}(f;x)=\sum_{k=0}^{n}f'(\xi_k){n\choose k}x^k(1-x)^{n-k},
\end{equation*}
където $\xi_k\in\left[\frac{k}{n+1}\frac{k+1}{n+1}\right], k=0,\ldots,n$.\\
\solution
Имаме, че 
\begin{equation*}
B_{n+1}(f;t)=\sum_{k=0}^{n+1}{n+1 \choose k} f\left(\frac{k}{n+1}\right)  t^{k-1}(1-t)^{n+1-k}.
\end{equation*}
Диференцираме по $t$.
\begin{align*}
B'_{n+1}(f;t) &= \sum_{k=1}^{n+1}{n+1 \choose k} f\left(\frac{k}{n+1}\right)kt_{k-1}(1-t)^{n+1-k} \\&+ \sum_{k=0}^{n}{n+1 \choose k} f\left(\frac{k}{n+1}\right)(n+1-k)t^k(1-t)^{n-k}(-1).
\end{align*}
Първата сума започва от $k=1$, защото за $к=0$ е 0, аналогично втората сума е до $k=n$.\\
За първата сума полагаме $j = k-1$.
\begin{align*}
B'_{n+1}(f;t) &= \sum_{j=0}^{n}{n+1 \choose j+1} f\left(\frac{j+1}{n+1}\right)(j+1)t_{j}(1-t)^{n-j} \\&- \sum_{k=0}^{n}{n+1 \choose k} f\left(\frac{k}{n+1}\right)(n+1-k)t^k(1-t)^{n-k} \\&=\sum_{k=0}^{n}{n+1 \choose k+1} f\left(\frac{k+1}{n+1}\right)(k+1)t_{k}(1-t)^{n-k} \\&- \sum_{k=0}^{n}{n+1 \choose k} f\left(\frac{k}{n+1}\right)(n+1-k)t^k(1-t)^{n-k} \\&=\sum_{k=0}^{n}t^k(1-t)^{n-k}\left({n+1 \choose k+1} f\left(\frac{k+1}{n+1}\right)(k+1)-{n+1 \choose k} f\left(\frac{k}{n+1}\right)(n+1-k)\right)\\&=\sum_{k=0}^n={n \choose k} t^k (1-t)^{n-k} (n+1)\left(f\left(\frac{k+1}{n+1}\right)-f\left(\frac{k}{n+1}\right)\right).
\end{align*}
От теоремата на Лагранж получаваме, че за някаква средна точка $\xi_k\in\left[\frac{k}{n+1}\frac{k+1}{n+1}\right]$ е изпълнено
\begin{equation*}
B'_{n+1}(f;t)=\sum_{k=0}^n={n \choose k} t^k (1-t)^{n-k} (n+1)f'(\xi_k)\left(\frac{k+1}{n+1}-\frac{k}{n+1}\right).
\end{equation*}
Откъдето
\begin{equation*}
B'_{n+1}(f;t)=\sum_{k=0}^n={n \choose k} t^k (1-t)^{n-k}f'(\xi_k)
\end{equation*}
\problem
Да се докаже, че полиномите на Льожандър 
\begin{equation*}
L_n(x) = \frac{1}{2^n n!}\left((x^2-1)^n\right)^{(n)}
\end{equation*}
удовлетворяват
\begin{equation*}
\int_{-1}^{1}L_n^2(x)dx=\frac{2}{2n+1}
\end{equation*} 
(Упътване: $L_n(1)=1, L_n(-1)=(-1)^n$.)\\
\solution
\begin{align*}
I=\int_{-1}^{1}L_n^2(x)dx = xL_n^2(x)\rvert_{-1}^1-2\int_{-1}^{1}L_n(x)L'(x)xdx.
\end{align*}
Имаме, че
\begin{align*}
L_n(x) = \alpha_nx^n + \ldots\\
L'_n(x)= n\alpha_nx^{n-1} + \ldots
\end{align*}

От $xL'_n(x)= n\alpha_nx^n+\ldots = nL_n(x) + P(x)$ следва, че
\begin{align*}
       I&= 2 - 2\int_{-1}^{1}L_n(x)(nL_n(x) + P)dx\\
        &= 2 - 2\int_{-1}^{1}nL_n^2(x)dx - \underbrace{2\int_{-1}^{1}\underbrace{L_n(x)}_{\text{четна ф-я}}\underbrace{P(x)dx}_{\text{нечетна ф-я}}}_{\text{интеграл на нечетна ф-я в симитричен интервал е 0}}\\
        &= 2 - 2nI
\end{align*}
Следователно $\boxed{I = \frac{2}{1+2n}}$.

\problem
\problem
\end{document}
